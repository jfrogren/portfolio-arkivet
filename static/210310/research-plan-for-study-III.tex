\documentclass[]{tufte-handout}

% ams
\usepackage{amssymb,amsmath}

\usepackage{ifxetex,ifluatex}
\usepackage{fixltx2e} % provides \textsubscript
\ifnum 0\ifxetex 1\fi\ifluatex 1\fi=0 % if pdftex
  \usepackage[T1]{fontenc}
  \usepackage[utf8]{inputenc}
\else % if luatex or xelatex
  \makeatletter
  \@ifpackageloaded{fontspec}{}{\usepackage{fontspec}}
  \makeatother
  \defaultfontfeatures{Ligatures=TeX,Scale=MatchLowercase}
  \makeatletter
  \@ifpackageloaded{soul}{
     \renewcommand\allcapsspacing[1]{{\addfontfeature{LetterSpace=15}#1}}
     \renewcommand\smallcapsspacing[1]{{\addfontfeature{LetterSpace=10}#1}}
   }{}
  \makeatother

\fi

% graphix
\usepackage{graphicx}
\setkeys{Gin}{width=\linewidth,totalheight=\textheight,keepaspectratio}

% booktabs
\usepackage{booktabs}

% url
\usepackage{url}

% hyperref
\usepackage{hyperref}

% units.
\usepackage{units}


\setcounter{secnumdepth}{-1}

% citations
\usepackage{natbib}
\bibliographystyle{plainnat}


% pandoc syntax highlighting

% longtable

% multiplecol
\usepackage{multicol}

% strikeout
\usepackage[normalem]{ulem}

% morefloats
\usepackage{morefloats}


% tightlist macro required by pandoc >= 1.14
\providecommand{\tightlist}{%
  \setlength{\itemsep}{0pt}\setlength{\parskip}{0pt}}

% title / author / date
\title[Forskningsplan för studie III]{Bostadsförsöket - en utvärdering}
\author{Joakim Frögren}
\date{2021-03-05}


\begin{document}

\maketitle




\begin{center}\rule{0.5\linewidth}{0.5pt}\end{center}

\hypertarget{titel}{%
\section{Titel}\label{titel}}

Bostadsförsöket - en utvärdering

\hypertarget{syfte-och-vetenskapliga-fruxe5gestuxe4llningar-1-2-meningar}{%
\section{Syfte och vetenskapliga frågeställningar (1-2
meningar)}\label{syfte-och-vetenskapliga-fruxe5gestuxe4llningar-1-2-meningar}}

Det övergripande syftet är att undersöka svenska seniorers syn på att
vara medforskare i ett nationellt så kallat `massexperiment' om
tillgänglighet i bostäder. Mer ingående handlar studien om att undersöka
vilket intresse och engagemang som finns hos denna grupp för att
medverka i forskning på detta sätt; vilken kapacitet äldre medborgare
tillskriver sig själva och uppvisar när det gäller att utföra de
uppgifter som ingår, samt vad de anser att detta engagemang kan ge dem i
form av t.ex. utökad kunskap om forskning, tillgänglighet i bostäder och
digital teknik. Avsikten med studien är ytterst att främja äldre
personers medverkan i forskning om åldrande och hälsa.

\hypertarget{omruxe5desuxf6versikt}{%
\section{Områdesöversikt}\label{omruxe5desuxf6versikt}}

\hypertarget{aktuellt-projekt-250-300-ord}{%
\subsection{Aktuellt projekt (250-300
ord)}\label{aktuellt-projekt-250-300-ord}}

Bostadsförsöket är ForskarFredags massexperiment 2021 och arrangeras av
föreningen Vetenskap \& Allmänhet i samarbete med forskare vid Centre
for Ageing and Supportive Environments (CASE) vid Lunds universitet.
Projektet syftar till att engagera skolungdomar, lärare och seniorer
till att vara medforskare genom att mäta tillgängligheten i den egna
bostaden med hjälp av tumstock (eller måttband) och en app på mobilen
(eller surfplattan).

Årets massexperiment motiveras av att det idag saknas giltiga och
tillförlitliga data om tillgängligheten i det svenska bostadsbeståndet
på nationell nivå. De försök till inventeringar som gjorts på kommunal
nivå är sinsemellan olika och därför kan data inte jämföras. De
nationella register som finns (Lägenhetsregistret, Fastighetsregistret)
har inga detaljerade uppgifter om förekomsten av miljöhinder. Det finns
ett flertal stora longitudinella forskningsprojekt om åldrandet, i
Sverige och internationellt, men de databaser som skapats innehåller
inga detaljerade data om bostäders tillgänglighet.

Tillgänglighet är bara en -- men viktig -- pusselbit i det komplexa
samspelet mellan den åldrande individen och miljön, men utgör dock en
första och basal förutsättning för att människor som åldras med
funktionsnedsättning ska kunna bo kvar i ''vanliga'' bostäder.
Bostadsförsöket utgör således en unik möjlighet att under en kort
tidsperiod erhålla en stor mängd data kring tillgänglighetsförhållanden
i det svenska bostadsbeståndet.

Utöver en unik möjlighet till masiv datainsamling erbjuder projektet
även möjligheten att undersöka hur deltagare uppfattar att delta i
forskning på detta sätt. Kunskap om äldre personers syn på att delta i
ett massexperiment kan användas för en utökad förståelse för vad som
krävs för att främja äldre personers medverkan i forskning om åldrande
och hälsa mer generellt. Kommunikation och översättning av kunskap är
centralt i projektet, med målet att uppnå största möjliga synlighet och
användning i resultaten kommer sådana aktiviteter att omfatta
vetenskapliga, samhälleliga, politiska och praktiska aspekter.

\hypertarget{bakgrund-200-250-ord}{%
\subsection{Bakgrund (200-250 ord)}\label{bakgrund-200-250-ord}}

Forskning saknas om seniorers deltagande i massexperiment. Trots att
citizen science eller medborgarforskning förekommit allt flitigare i
forskningslitteraturen inom åldrande och hälsa de senaste decennierna,
adresseras sällan specifikt massexperiment som metod. Massexperiment har
en stor potential inte minst eftersom metoden möjliggör datainsamling av
en storleksordning som professionella forskare har svårt att själva
uppbåda. Involverandet av äldre personer spelar en betydelsefull roll
inom forskningen om åldrande och hälsa genom den kunskap och erfarenhet
de besitter om att åldras och hur åldrandet relaterar till hälsa (ref).
Flera studier handlar om hur en bredare allmänhet engageras i
massexperiment. Få studier uppmärksammar specifikt äldre personer
massexperiment (ref). Därför kommer vi att studera seniorers syn på att
delta i forskning på detta sätt ur ett brett, explorativt perspektiv som
omfattar såväl vilket intresse och engagemang för detta slags
deltagande, kapacitet för detta åtagande samt vad seniorer anser att de
får ut av det.

\hypertarget{projektbeskrivning}{%
\subsection{Projektbeskrivning}\label{projektbeskrivning}}

\hypertarget{design-60-70-ord}{%
\subsubsection{Design (60-70 ord)}\label{design-60-70-ord}}

Den aktuella studien har en kvantitativ design med data insamlad via en
enkät med grupper utvalda utefter kriterium att de är 67 år eller äldre
samt har en varierande boendeform. \#\#\# Rekrytering och urval (250
ord) Deltagarna kommer att väljas efter att de är 67 år eller äldre samt
har en variead boendeform. Åldersgruppen stratifieras på kön. Baserat på
tidigare erfarenheter av enkätstudier genomförda i Sverige förväntar vi
oss en svarsfrekvens på 50-60\% (Ryan et al.~2016). Mot denna bakgrund
och för att ha ett tillräckligt populationsunderlag kommer vi att
använda {[}beskriv{]} strategin{]}. Rekrytering av deltagare ur detta
underlag kommer att ske till att X personer i tills dess att X personer
med bostadsform Y och Z personer med bostadsform Q har besvarat enkäten.
Deltagarna kommer att inkluderas efter informerat samtycke, i enlighet
med bilaga (nr X). Samtliga deltagare kommer att tilldelas ett
löpnummer.

\hypertarget{procedurgenomfuxf6rande-400-500-ord}{%
\subsubsection{Procedur/genomförande (400-500
ord)}\label{procedurgenomfuxf6rande-400-500-ord}}

För studiens syfte har en enkät utformats av forskargruppen (Bilaga X),
baserad på befintlig forskningslitteratur. Enkäten kommer ordagrant att
omvandlas till ett webbformulär för att kunna besvaras online.
Formuläret kommer att publiceras på säker webbplats i enlighet med
General Data Protection Regulation. Innan den fullständiga
datainsamlingen påbörjas kommer forskargruppen genomföra en pilotstudie
med X deltagare. Pilotstudien kommer att genomföras med X personer ur
CASE brukarråd som anses utgöra ett urval av den internetanvändande
gruppen personer 60 år och äldre i Sverige. Eventuell revidering av
enkät genomförs av forskargruppen innan den fullständiga datainsamlingen
påbörjas.

Deltagarna kommer att kontaktas av X genom e-post/brev, för informerat
samtycke enligt bilaga X. I brevet hänvisas till en webbversion av
enkäten och deltagarna tilldelas unika inloggningsuppgifter.

En första påminnelse skickas mejlledes till deltagarna efter {[}x{]}
vecka/veckor, tillsammans med informationsbrev (Bilaga X), hänvisning
till webbenkät och inloggningsuppgifter. Om svar uteblir från deltagare
kommer efter 2-3 veckor en påminnelse om att besvara webbenkäten att gå
ut.

Kvalitetskontroll för att säkerställa att frågorna inte missförstås, att
data överensstämmer med deltagarnas svar och är korrekt, fullständigt
samt logiskt registrerade i databasen, genomförs av X. Eventuella
ändringar som behöver göras för att säkerställa detta kommuniceras med
forskargruppen. Datainsamlingen kommer även att göras tillgänglig för
monitorering av forskare från forskargruppen enligt följande: när
{[}x\%{]} enkäterna besvarats granskas data för att identifiera
eventuella systematiska fel i datainsamlingen.

\hypertarget{analys-1-2-meningar}{%
\subsubsection{Analys (1-2 meningar)}\label{analys-1-2-meningar}}

Data kommer att analyseras med vedertagna metoder samt med hänsyn till
typ av data (kvalitativ respektive kvantitativ, normalfördelning, etc.)
inklusive skalnivå.

\hypertarget{hantering-av-insamlad-data-250-ord}{%
\subsubsection{Hantering av insamlad data (250
ord)}\label{hantering-av-insamlad-data-250-ord}}

Ett krypterat system kommer att användas i samband med datainsamlingen
för att garantera att de uppgifter som deltagarna lämnar i enkäten
hanteras konfidentiellt i enlighet med GDPR. Forskargruppens
datahanteringsansvarig, dr. X, kommer att avidentifiera deltagarnas svar
på enkätfrågorna så att övriga artikelförfattare inte får tillgång till
identifierande uppgifter såsom namn, personnummer eller adress. Ålder
samt postnummer för hemadress kommer dock att ingå i databasen för att
kunna iaktta eventuell betydelse av ålder eller bostadsort för
resultaten. När resultaten rapporteras kommer således inga enskilda
individer kunna identifieras.

Data kommer att lagras i avidentifierat skick vid i en databas som
endast forskare som är involverade i projektet har åtkomst till. För att
skydda data i projektet mot obehörig åtkomst kommer plattformen LUSEC
vid, Medicinska fakulteten, Lunds universitet att användas. I LUSEC har
data stark kryptering och är endast åtkomliga efter
tvåfaktorautentisering och selektiv behörighetskontroll. {[}Person X{]}
kommer på delegation av projektansvarig (professor Susanne Iwarsson) att
administrera vilka forskare som ges behörighet att komma åt data i
LUSEC. Materialet kommer att arkiveras under 10 år. Ingen journalföring
förekommer eftersom projektet inte har någon koppling till hälso- och
sjukvårdsinsatser.

\hypertarget{redoguxf6relser-fuxf6r-tidigare-erfarenheter-av-vald-metod-60-70-ord}{%
\subsubsection{Redogörelser för tidigare erfarenheter av vald metod
(60-70
ord)}\label{redoguxf6relser-fuxf6r-tidigare-erfarenheter-av-vald-metod-60-70-ord}}

Genomförande av enkätstudier är vanligt förekommande inom medicinsk- och
hälsovetenskaplig forskning och anses inte medföra några egentliga
risker för de som medverkar. Projektet drivs av seniora forskare (SI,
MG) med lång erfarenhet av forskning som bygger på frågeformulär och
självskattningsinstrument som resulterat i ett stort antal
originalartiklar publicerade i välrenommerade internationella
tidskrifter.

\hypertarget{forskningsetiska-uxf6vervuxe4ganden-200-ord}{%
\subsubsection{Forskningsetiska överväganden (200
ord)}\label{forskningsetiska-uxf6vervuxe4ganden-200-ord}}

Potentiella deltagare kommer att erhålla muntlig information om studien
och ge sitt informerade samtycke i samband med datainsamlingen genom att
besvara enkäten. Det finns även möjlighet att kontakta forskarna för
ytterligare information inför informerat samtycke. Information lämnas om
att allt deltagande i studien är frivilligt och att det finns möjlighet
att när som helst avbryta sitt deltagande. Studien bedöms inte innebära
någon risk för skada eller obehag för deltagarna. Enkäten beräknas ta
10-15 minuter att besvara, vilket anses vara en rimlig tid som inte
kommer att belasta deltagarna nämnvärt. Deltagarna kommer att informeras
om att de när som helst kan avbryta eller välja att inte svara på
specifika frågor utan att uppge någon anledning. Deltagandet förväntas
inte att innebära några för- eller nackdelar för den enskilde individen.
Alla resultat kommer att redovisas i avidentifierat skick så att inga
resultat kommer att kunna knytas till enskilda individer.

Resultaten kommer att ingå i en doktorsavhandling och spridas till andra
forskare vid vetenskapliga konferenser och publicering i välrenommerade
internationella vetenskapliga tidskrifter. Projektet är finansierat av X
och deras krav på open access kommer att efterlevas. Vetenskaplig
publicering förväntas ske under 2021-2022. Resultaten kommer även att
kommuniceras populärvetenskapligt.

\hypertarget{studiens-betydelse-250-300-ord}{%
\subsubsection{Studiens betydelse (250-300
ord)}\label{studiens-betydelse-250-300-ord}}

Forskning saknas om {[}fenomenet i fråga{]}. Få studier har övergripande
fokuserat på åldrande och {[}x{]} med särskild uppmärksamhet på {[}y{]}
och {[}z{]}. Genom ökad kunskap om {[}x{]} kan samhället {[}eller annan
instans/aktör{]} på ett bättre sätt möta {[}y{]}:s faktiska behov. Ökad
kunskap om {[}x{]}:s attityder och erfarenheter, behov och önskemål
kommer att få betydelse för {[}y{]} Denna studie utgör det tredje steget
i ett större projekt som syftar till att z\ldots{} och utgör en viktig
grund för att studera området ur ett bredare befolkningsperspektiv. Att
förstå x och dess betydelse för y bidrar till projektets
samhällsrelevans liksom till potentialen för gränsöverskridande
forskning inom detta område. Vi kommer också att rapportera om det finns
några könsskillnader som är värda att uppmärksamma. Resultaten kan bidra
till utvecklingen av hälsofrämjande och förebyggande insatser på
individ, grupp och samhällsnivå. Studien kan ligga till grund för
utbildning av yrkesgrupper inom hälso- och sjukvård och socialtjänst,
inklusive utveckling av insatser i hemmet för att stödja aktivt och
hälsosamt åldrande.

Kommunikation och översättning av kunskap är centralt i projektet. Med
målet att uppnå största möjliga synlighet och användning av resultaten
kommer sådana aktiviteter att omfatta vetenskapliga, samhälleliga,
politiska och praktiska aspekter.

\hypertarget{tidplan}{%
\subsubsection{Tidplan}\label{tidplan}}

\hypertarget{rekrytering-av-deltagare}{%
\paragraph{Rekrytering av deltagare}\label{rekrytering-av-deltagare}}

Rekrytering av deltagare kommer att ske med start\ldots{}

\hypertarget{genomfuxf6randet-av-studien}{%
\paragraph{Genomförandet av studien}\label{genomfuxf6randet-av-studien}}

\hypertarget{fuxf6renkuxe4ten}{%
\subparagraph{Förenkäten}\label{fuxf6renkuxe4ten}}

Förenkäten kommer att skickas ut till de föreningar som anmält sitt
intresse för att delta under andra halvan av augusti 2021

\hypertarget{efterenkuxe4ten}{%
\subparagraph{Efterenkäten}\label{efterenkuxe4ten}}

Efterenkäten kommer att skickas ut till de föreningar som anmält sitt
intresse för att delta under första halvan av oktober 2021

\hypertarget{analys}{%
\paragraph{Analys}\label{analys}}

Analys av materialet kommer att påbörjas under andra halvan av oktober
2021

\hypertarget{enkuxe4ten}{%
\section{Enkäten}\label{enkuxe4ten}}

\hypertarget{titel-1}{%
\subsection{Titel}\label{titel-1}}

Bostadsförsöket - en utvärdering

\hypertarget{brevets-formulering}{%
\subsection{Brevets formulering}\label{brevets-formulering}}

Hej,

Vill du delta i en utvärderingsstudie om Bostadsförsöket? I detta brev
får du information om utvärderingsstudien och vad det innebär att delta.
{[}3-4 meningar om rationale{]} Tanken med forskningsprojektet X är att
utveckla ny kunskap om\ldots{} I forskningsprojektet vill vi även
undersöka X. {[}1-2 förklarande meningar{]}. Utifrån dina och andra
deltagares svar vill vi utveckla ny kunskap om hur\ldots{}

\hypertarget{hur-guxe5r-studien-till}{%
\subsubsection{Hur går studien till?}\label{hur-guxe5r-studien-till}}

Om du väljer att delta kommer du att få besvara en webbenkät som det tar
ungefär 10-15 minuter att besvara. Enkäten skickas ut till cirka
{[}antal{]} personer som är..{[}urvalskriterie{]}.

\hypertarget{vad-huxe4nder-med-mina-uppgifter}{%
\subsubsection{Vad händer med mina
uppgifter?}\label{vad-huxe4nder-med-mina-uppgifter}}

Datainsamlingen genomförs av forskare inom Centre for Ageing and
Supportive Environments (CASE) vid Lunds universitet. Våra
kontaktuppgifter hittar du längst ner i detta brev.

X har fått dina adressuppgifter genom {[}beskriv hur{]} från X:s
adressregister. Dina svar i enkäten lagras på ett säker sätt av
{[}beskriv av vem{]}. X kommer dock inte att behålla något av
materialet, utan överlämna dina enkätsvar avidentifierade, det vill säga
utan ditt namn, personnummer eller adress, till forskarna. Dina svar
kommer bara att användas av forskare för att besvara studiens syfte.
Materialet kommer att arkiveras i tio år och i enlighet med EU:s
dataskyddsförordning, General Data Protection Regulation (GDPR), i en
säker portal som bara projektets forskare har tillgång till.

\hypertarget{hur-fuxe5r-jag-information-om-resultatet-av-studien}{%
\subsubsection{Hur får jag information om resultatet av
studien?}\label{hur-fuxe5r-jag-information-om-resultatet-av-studien}}

Resultatet kommer att presenteras på gruppnivå i vetenskapliga och
populärvetenskapliga sammanhang. Ingen utomstående kan komma att
identifiera dig. Studien kommer också att ingå i en doktorsavhandling.

\hypertarget{deltagandet-uxe4r-frivilligt}{%
\subsubsection{Deltagandet är
frivilligt}\label{deltagandet-uxe4r-frivilligt}}

Ditt deltagande i studien är helt frivilligt och innebär inget
ytterligare åtagande. Du kan också avstå från att besvara enstaka frågor
och kan när som helst avbryta ditt deltagande. Om du väljer att delta så
är nästa steg att fylla i enkäten som du når via länken nedan. Om du
inte vill delta behöver du inte göra något, men en första och en andra
påminnelse kommer att skickas per {[}mail/brev{]} till personer som inte
besvarat enkäten. Du kan besvara enkäten via nedanstående länk som du
skriver in i webbläsaren på din dator eller surfplatta.

Länk: \url{https://sifologin.tns-sifo.se} Användarnamn: {[}här anges ett
användarnamn som är unikt för varje deltagare{]} Lösenord: {[}här anges
ett lösenord som är unikt för varje deltagare{]}

\hypertarget{om-du-undrar-nuxe5got}{%
\subsubsection{Om du undrar något}\label{om-du-undrar-nuxe5got}}

Har du några frågor om undersökningen är du välkommen att kontakta
forskare vid Lunds universitet:Anna Andersson: via telefon
{[}telefonnummer{]}, eller e-post {[}e-postadress{]}.

\hypertarget{ansvarig-fuxf6r-studien}{%
\subsubsection{Ansvarig för studien}\label{ansvarig-fuxf6r-studien}}

Forskningshuvudman är Institutionen för hälsovetenskaper, Medicinska
fakulteten vid Lunds universitet. Samtliga forskare i denna studie är
verksamma där. Vetenskapligt ansvarig forskare för studien är:

Susanne Iwarsson, professor i gerontologi\\
Telefon: 046-222 19 40\\
E-post:
\href{mailto:susanne.iwarsson@med.lu.se}{\nolinkurl{susanne.iwarsson@med.lu.se}}

\hypertarget{enkuxe4t-fuxf6r-bostadsfuxf6rsuxf6ket---en-utvuxe4rdering}{%
\subsection{Enkät för Bostadsförsöket - en
utvärdering}\label{enkuxe4t-fuxf6r-bostadsfuxf6rsuxf6ket---en-utvuxe4rdering}}

I denna enkät finns frågor om {[}en mening om innehåll{]}. Inga svar är
rätt eller fel utan vi är intresserade av din uppfattning kring dessa
frågor.

\hypertarget{process-och-genomfuxf6rbarhet-involvering-och-support}{%
\subsubsection{Process och genomförbarhet: Involvering och
support}\label{process-och-genomfuxf6rbarhet-involvering-och-support}}

\hypertarget{muxe5lgruppsinriktning}{%
\paragraph{Målgruppsinriktning}\label{muxe5lgruppsinriktning}}

\begin{itemize}
\tightlist
\item
  Hur väl skulle du säga att arrangörerna av detta evenemang har lyckats
  kommunicera/ förklara detta projekt på ett tillräckligt tydligt sätt
  för dig? (Mycket väl - Inte alls väl)
\item
  Om du inte är nöjd med denna kommunikation, vad kunde ha gjorts
  annorlunda för att förbättra den? Fler alternativ är möjliga (a)
  Använt andra kommunikationskanaler; b) Formulerat innehållet
  annorlunda; c) Kommunicerat mer frekvent; d) Annat, nämligen\ldots)
\item
  I vilken utsträckning är `Bostadsförsöket' ett projekt som väcker
  engagemang hos dig? (I mycket hög grad - I väldigt liten grad)
\item
  Om inte `Bostadsförsöket' väcker så stort engagemang hos dig, vad
  beror det på? (a) Ämnet som sådant intresserar mig inte tillräckligt
  b) Projektet såsom det är utformat är inte tillräckligt intressant c)
  Jag har inte tid eller ork att engagera mig i det d) Annat,
  nämligen\ldots{}
\end{itemize}

\hypertarget{intensitetsgrad}{%
\paragraph{Intensitetsgrad}\label{intensitetsgrad}}

\begin{itemize}
\tightlist
\item
  Skulle du ha velat delta i fler faser av detta projekt, t.ex. i
  planeringen av Bostadsförsöket? (Ja / Nej / Jag vet inte)
\item
  Om ja, varför? Flera svar möjliga (a) För att på så vis lära mig mer
  om hur man bedriver forskning? b) För att lära mig mer om
  tillgänglighet i bostäder c) För att lära mig mer om teknik, t.ex.
  utveckling av appar)
\item
  Om nej, varför inte? Flera svar möjliga (a) Inte tillräckligt
  intresserad av forskning b) Inte tillräckligt intresserad av
  tillgänglighet i bostäder c) Inte tillräckligt intresserad av teknik,
  t.ex. utveckling av appar d) Jag har inte tid eller ork att engagera
  mig ytterligare
\item
  Vilket påstående tycker du bäst beskriver din roll i detta projekt i
  förhållande till de professionella forskarnas roll när det gäller att
  generera kunskap? a)
\end{itemize}

\hypertarget{underluxe4ttande-och-kommunikation}{%
\paragraph{Underlättande och
kommunikation}\label{underluxe4ttande-och-kommunikation}}

(Enkät på Likert-skala, allt från 1 (helt oense) till 5 (helt överens) -
Stödet till detta projekt har anpassats tillräckligt till dig och din
kapacitet? - Utbildningsåtgärderna för detta projekt har anpassats
tillräckligt till dig och din kapacitet? - Målen för detta projekt har
kommunicerats på ett tillräckligt tydligt och öppet sätt till dig? -
Resultaten av detta projekt har kommunicerats på ett tillräckligt
tydligt och transparent sätt till dig?

\begin{itemize}
\tightlist
\item
  Hur har du uppfattat forskarnas sätt att interagerande med dig under
  detta projekt? (utmärkt - mycket bristfälligt)
\item
  Skulle du ha velat interagera mer med forskarna ? (ja/nej/vet inte)
\end{itemize}

\hypertarget{samarbete-och-synergier}{%
\paragraph{Samarbete och synergier}\label{samarbete-och-synergier}}

\begin{itemize}
\tightlist
\item
  Hur fick du information om möjligheten att delta i `Bostadsförsöket'?
  (a) genom mejl från intresseförening, b)
\item
  Om du hört talas om det från en organisation / förening som du är
  medlem i, anser du att informationen där kommunicerades på ett
  tillräckligt tydligt och öppet sätt till dig? (ja/nej/vet inte)
\end{itemize}

\hypertarget{resultat-och-puxe5verkan-individuell-utveckling}{%
\subsubsection{Resultat och påverkan: Individuell
utveckling}\label{resultat-och-puxe5verkan-individuell-utveckling}}

\hypertarget{kunskap-fuxe4rdigheter-kompetenser}{%
\paragraph{Kunskap, färdigheter,
kompetenser}\label{kunskap-fuxe4rdigheter-kompetenser}}

\begin{itemize}
\tightlist
\item
  Vilka är dina mål / vad hoppas du uppnå genom att delta i
  Bostadsförsöket? (a) Lära mig mer om forskning; b) Lära mig mer om
  tillgänglighet i bostäder; c) Lära mig mer om digital teknik; d)
  Annat, nämligen\ldots{}
\item
  (FÖRE: Tror du att du kommer erhålla / EFTER: Erhöll du ..) ny
  kunskap, nya färdigheter eller nya kompetenser relaterade till
  forskning genom att delta i `Bostadsförsöket'?
\item
  (FÖRE: Tror du att du kommer erhålla / EFTER: Erhöll du ..) ny
  kunskap, nya färdigheter eller nya kompetenser relaterade till
  bostadstillgänglighet genom att delta i `Bostadsförsöket'?
\item
  (FÖRE: Tror du att du kommer erhålla/ EFTER: Erhöll du ..) ny kunskap,
  nya färdigheter eller nya kompetenser relaterade till informations-
  och kommunikationsteknik (IKT) från att delta i Bostadsförsöket?
\item
  (FÖRE: Tror du att du kommer erhålla/ EFTER: Erhöll du ..) ny kunskap,
  färdigheter eller kompetenser relaterade till något annat från att
  delta i `Bostadsförsöket'?

  \begin{itemize}
  \tightlist
  \item
    Om ja, ange vad \ldots{}
  \end{itemize}
\end{itemize}

\hypertarget{attityder-och-vuxe4rden}{%
\paragraph{Attityder och värden}\label{attityder-och-vuxe4rden}}

\begin{itemize}
\tightlist
\item
  (FÖRE: Tror du att ditt / EFTER: Uppfattar du att ditt\ldots)
  deltagande i `Bostadsförsöket' (kommer att ha / har haft) påverkan på
  dina värderingar och attityder angående vetenskap / forskning?
\item
  Om ja, på vilket sätt? (a) Mer postivt inställd; b) Mer negativt
  inställd; c) På annat sätt, nämligen\ldots{}
\end{itemize}

\hypertarget{beteende-och-uxe4gande}{%
\paragraph{Beteende och ägande}\label{beteende-och-uxe4gande}}

\begin{itemize}
\tightlist
\item
  Hur mycket engagemang och ansvar skulle du säga att du har avkrävts av
  arrangörerna av detta evenemang? (Väldigt mycket - Vädigt liten)
\item
  Skulle du säga att `Bostadsförsöket' främjar `ownership', d.v.s.
  främjar en slags vilja och förmåga att vidta åtgärder för att hantera
  de problem som står på spel, såsom tillgänglighetsproblem i ditt hem?
  (Ja/ Nej/Vet inte)
\item
  (FÖRE: Tror du att `Bostadsförsket' kommer att bidra \ldots{} / EFTER:
  Har `Bostadsförsöket' bidragit ..) till personlig förändring i ditt
  beteende?

  \begin{itemize}
  \tightlist
  \item
    På vilket sätt? \ldots{}
  \end{itemize}
\end{itemize}

\hypertarget{motivationer-och-engagemang}{%
\paragraph{Motivationer och
engagemang}\label{motivationer-och-engagemang}}

\begin{itemize}
\tightlist
\item
  (FÖRE: Tror du att ditt / EFTER: Uppfattar du att ditt\ldots)
  deltagande i `Bostadsförsöket' (kommer att höja/ har höjt) din
  motivation?

  \begin{itemize}
  \tightlist
  \item
    Om ja, på vilket sätt / sätt?
  \end{itemize}
\item
  (FÖRE: Tror du att ditt / EFTER: Uppfattar du att ditt\ldots)
  deltagande i `Bostadsförsöket' Kan höja/har hänt att din självkänsla?

  \begin{itemize}
  \tightlist
  \item
    (Tror du att deltagande i `Bostadsförsöket' kommer att motivera dig
    ../ Är du motiverad ..) att fortsätta projektet eller delta i
    liknande aktiviteter?
  \end{itemize}
\end{itemize}

\hypertarget{demografi}{%
\subsection{Demografi}\label{demografi}}

\begin{itemize}
\tightlist
\item
  (Ålder) Vad är din ålder? (\ldots{} år)
\item
  (Kön) Vad identifierar du dig som? (Man/Kvinna/Övrigt)
\item
  (Utbildning) Vilken är din högsta utbildning? (Grundskola
  (folkskola/flickskola/realskola)/Gymnasium (folkhögskola)/
  Eftergymnasial utbildning, mindre än tre år
  (yrkeshögskola/kvalificerad yrkesutbildning)/ Eftergymnasial
  utbildning, tre år eller mer/ Forskarutbildning/ Annat/Vill ej uppge)
\item
  Hur ser din boendesituation ut? (Bor själv eller är särbo/Är
  sambo/Övrigt)
\item
  (Typ av bostad) (Villa/Radhus/Lägenhet/Servicelägenhet/Särskilt
  boende/Fritidsboende/Gård/Övrigt)
\item
  Hur länge har du bott i din nuvarande bostad?
\item
  Etnicitet
\item
  (Ekonomi) Vilket av följande alternativ tycker du själv bäst beskriver
  din ekonomiska situation? (Dålig/Någorlunda/God/Mycket god/Utmärkt)
\end{itemize}

Eftersom enkäten berör forskning om åldrande och hälsa handlar frågorna
26--30 om ditt hälsotillstånd. Dina svar på hälsofrågorna kommer att ge
forskarna ökade möjligheter att göra gruppjämförelser.

\hypertarget{huxe4lsa}{%
\subsection{Hälsa}\label{huxe4lsa}}

\begin{itemize}
\tightlist
\item
  Instrument: Självskattad hälsa) I allmänhet skulle du säga att din
  hälsa är(markera ett av alternativen): (Dålig/Någorlunda/God/Mycket
  god/Utmärkt)
\end{itemize}

\hypertarget{skuxf6rhet}{%
\subsection{Skörhet}\label{skuxf6rhet}}

\begin{itemize}
\tightlist
\item
  Orkar du gå en promenad på cirka 15--20 minuter? (Ja/Nej)
\item
  Har du känt dig allmänt trött eller upplevt nedsatt ork desenaste tre
  månaderna? (Ja/ Nej)
\item
  Upplever du att du ramlar ofta eller är rädd för att ramla? (Ja/Nej)
\item
  Behöver du hjälp med att handla, det vill säga att ta dig till
  affären, plocka varor, betala och bära hem varorna? (Ja/Nej)
\end{itemize}

\hypertarget{ict-kunskaperfarenhet}{%
\subsection{ICT-kunskap/erfarenhet}\label{ict-kunskaperfarenhet}}

\hypertarget{study-specific-questions}{%
\subsubsection{Study-specific
questions}\label{study-specific-questions}}

\begin{itemize}
\tightlist
\item
  On average how often would you say you have been using a smartphone or
  tablet during the last 3 months?
\item
  How often do you use the internet on your smartphone or tablet?
\item
  How knowledgeable do you consider yourself when it comes to using a
  smartphone or a tablet?
\end{itemize}

\hypertarget{instrument-for-measuring-older-peoples-attitudes-towards-technology}{%
\subsubsection{Instrument for Measuring Older People's Attitudes Towards
Technology}\label{instrument-for-measuring-older-peoples-attitudes-towards-technology}}

\url{https://bth.diva-portal.org/smash/get/diva2:1324996/FULLTEXT01.pdf}
(Likert scale questionnaire, ranging from 1 (fully disagree) to 5 (fully
agree))

\begin{enumerate}
\def\labelenumi{\arabic{enumi}.}
\tightlist
\item
  I think it's fun with new technological gadgets
\item
  Using technology makes life easier for me
\item
  I like to acquire the latest models or updates
\item
  I am sometimes afraid of not being able to use the new technical
  things
\item
  Today, the technological progress is so fast that it's hard to keep up
\item
  I would have dared to try new technical gadgets to a greater extent if
  I had had more support and help than Ihave today
\item
  People who do not have access to the internet have a real disadvantage
  because of all that they are missingout on
\item
  Too much technology makes society vulnerable
\end{enumerate}

\hypertarget{tidigare-kunskaperfarenhet-av-tillguxe4nglighet-i-bostuxe4der}{%
\subsection{Tidigare kunskap/erfarenhet av tillgänglighet i
bostäder}\label{tidigare-kunskaperfarenhet-av-tillguxe4nglighet-i-bostuxe4der}}

\begin{itemize}
\tightlist
\item
  Jag har i min profession arbetat med frågor som rör tillgänglighet i
  bostäder (Ja/Nej/Vet inte)
\item
  Om ja, vilken var din roll? (stadsplanerare/ arkitekt/
  byggnadsingenjör/ byggnadsarbetare/ fastighetsförvaltare/ forskare/
  övrigt)
\item
  Jag har under min fritid engagerat mig i frågor som rör tillgänglighet
  i bostäder (Ja/ Nej / Vet inte)
\item
  Om ja, på vilket sätt? (Genom engagemang i en intresseförening/ Genom
  engagemang i en bostadsrättsförening/ Genom partipolitiskt engagemang/
  Övrigt)
\end{itemize}

\bibliography{library.bib}



\end{document}
